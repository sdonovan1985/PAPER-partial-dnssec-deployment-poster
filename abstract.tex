\begin{abstract}
DNSSEC has been in development for 20 years. It provides for provable security
when retrieving domain names through the use of a public key infrastructure 
(PKI). Unfortunately, there is also significant overhead involved with DNSSEC:
verifying certificate chains of signed DNS messages involves extra computation,
queries to remote resolvers, additional transfers, and introduces added latency
into the DNS query path.
We pose the question: is it possible to achieve practical security without 
always verifying this certificate chain if we use a different, outside source of
trust between resolvers?
We believe we can. 
Namely, by using a long-lived, mutually authenticated
TLS connection between pairs of DNS resolvers, we suggest that we can maintain 
near-equivalent levels of security with very little extra overhead compared to a
non-DNSSEC enabled resolver. 
By using a reputation system or probabilistically verifying a portion of DNSSEC
responses would allow for near-equivalent levels of security to be reached, even
in the face of compromised resolvers.

\end{abstract}
