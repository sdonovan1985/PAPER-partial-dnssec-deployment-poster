\section{Conclusion and Future Work}\label{sec:conclusion}

We have proposed using a long-lived, mutually authenticated TLS connection
between two DNSSEC-enabled resolvers as an alternative source of trust in order
to reduce the overhead of performing DNSSEC signature  verification operations.
We have shown that, while having potential security issues, we can mitigate 
compromised machines by probabilistically verifying DNSSEC responses to build 
trust.

This method is similar to how stub resolvers (\ie, hosts) maintain DNSSEC 
guarantees. RFC 4033 specifically suggests using IPSEC to maintain security
between the stub resolver and the recursive resolver.\cite{rfc4033} Using TLS
provides similar guarantees. Additionally, this method expands the ongoing work
of the DNS privacy working group within the IETF from not only stub-to-recursive
resolver connections, but between recursive resolvers.

As the security of this proposal hinges on the trustworthiness of the DNS 
resolver that a TLS connection is established with, our first course of action 
is to create and simulate our proposed solutions for establishing trust. 
While some scenarios could be represented by a Bayesian model, others lend 
themselves to simulation. In particular, as the probabilities for verification 
change based on previous input, simulation is easier for determining whether a
a particular strategy for verification is effective.

Developing models for malicious resolver behavior and strategies for countering


These
need not involve actual DNS resolution, rather these can be simulated in a much
simpler environment. Some scenarios may be modeled using Bayes rules, but only
if probabilities are known and consistent. Other scenarios have ``moving 
parts'' --- such as changes in the probability of verification --- where 
simulation is more appropriate. Finally, different behaviors of a malicious 
resolver can be more easily simulated, such as a resolver that only `lies' about
one particular domain and provides valid results for all others. Different 
strategies may be necessary for establishing trust 

Further, we plan on implementing this scheme on an existing 
DNSSEC-enabled resolver, such as BIND or Unbound. 
We will then test this method for both
practicality --- verifying that there is a processing and network  cost savings 
compared to verifying all DNSSEC messages --- and security --- by `compromising'
one resolver and seeing the response from the non-compromised resolver, 
specifically how long it takes to find that the one resolver is ``lying''.

Further, we see this alternative model of trust could be used in other 
scenarios. For instance, the various security schemes surrounding BGP (BGPSEC, 
S-BGP, RPKI, etc.) all rely on PKI verification, much like DNSSEC. This 
technique could be applied for similar savings with potentially different 
criteria for verification, such as for any prefixes being announced by 
autonomous systems (ASes) known to host malware. 

This is not a method for absolute security, like DNSSEC is aiming for, rather 
this is for practical security with low overhead. It can be used to alleviate
scalability concerns particularly on a closed network where all DNS resolvers are
controlled by the same organization. It is also useful for an organization that
has less strict security requirements but wants to benefit from DNSSEC with 
lower cost.

