\section{Proposal}\label{sec:proposal}

We want to reduce the overhead of the certificate chain verification in DNSSEC.
By reducing the number of signature  verifications that are necessary, we reduce
the majority of overhead costs --- queries, bandwidth, and latency --- to near 
zero. \footnote{DNS responses will remain larger due to the addition of  DNSSEC
signature records which are not present in vanilla DNS.}

To do this, we propose using a long-lived, mutually authenticated TLS connection
between resolvers. With mutual authentication, both resolvers can provably show
that they know who they are talking to. If one resolver trusts what the  other 
resolver says, there should not be a need to re-verify the DNSSEC signature
chain. This reduces the overhead to near zero, as there will only be
DNSSEC verification if the remote resolver does not use our proposed trust
mechanism.

There are two questions that are still left to answer. First, how do we know a
DNSSEC resolver is trustworthy? Second, what if the remote resolver has been 
compromised? These are both variants of the same question: How can we show that
the remote resolver is trustworthy?
To solve this final problem, we have two solutions. The first solution is
entirely internal to the resolver trying to establish trust: verifying a 
portion of the DNS responses from the remote resolver. If the local resolver 
verifies, say, 5\% of all responses for validity, it probabilistically confirms 
that the remote resolver is trustworthy. This will reduce the amount of extra 
work (over traditional DNS) by near 95\%. This number can be adjusted up or 
down, depending on level of trust (or paranoia) that the administrator has in
the operator of the remote resolver. One automated method would be to perform
DNSSEC certificate chain verification on all responses at the beginning of a 
session, then slowly tapering down to a much lower level of only a few percent
after a while without issue.

The second solution involves using a reputation system, which are typically 
external to a domain. If a large enough number of resolvers use the 
probabilistic system described above, a reputation system could be established 
by aggregating whether or not a particular resolver has ``lied'' and provided 
incorrect DNSSEC responses. 
Other resolvers that use the reputation system could tailor their 
verification strategy based on reputation of the remote resolver. 
For instance, one resolver that cares very strongly about 
security could verify any and all responses from resolvers that it uses, and 
report on the other resolvers' trustworthiness.
A more security lax resolver may verify 50\% of responses from resolvers with a
``bad'' reputation, but only 5\% of responses from those with a ``good'' 
reputation, just to keep them honest and to report back to the reputation 
system. 

These are two \emph{classes} of solutions. There can be  multiple different 
different versions, each tailored to different behaviors of malicious resolvers.
Finding a middle ground to handle different types of behavior well, while 
reducing DNSSEC signature verification overhead is the main challenge.

There is a certificate chain verification cost when establishing a TLS 
connection, but it is a one-time cost, rather the constant recurrent cost like 
DNSSEC signature verifications. The overhead reductions can be seen in 
Table~\ref{table:costs}. The overhead reduction is primarily dependent on how 
much DNSSEC signature verification an administrator wants to use. It is also 
dependent on how long lived the TLS connections are. Perfect reduction is not 
possible while still maintaining reasonable security, if only  because multiple 
connections may be needed for different remote resolvers.

\begin{table}
  \centering
  \begin{tabular}{|l|l|l|}
    \hline
    & \textbf{PKI Ops.} & \begin{tabular}{l}
      \textbf{Additional} \\ 
      \textbf{Queries} \end{tabular} \\
    \hline
    \textbf{Traditional DNS} & 0 & 0 \\
    \hline
    \textbf{DNSSEC} & $N$ & $\sim10N$ \\
    \hline
    \textbf{With Alt. Trust} & 1 & 0 \\
    \hline
    \textbf{5\% Verified Alt. Trust} & $1+\sim0.05N$ & $\sim0.5N$ \\
    \hline
  \end{tabular}
  \caption{Expected overhead versus traditional DNS. \emph{Alt. Trust} refers to
    our proposed solution without any additional verification. 
    \emph{5\% Verified Alt. Trust} refers to verifying 5\% of ``trusted'' 
    responses from other DNS resolver. Transfers are estimated as a factor of 10
    over traditional DNS.\cite{huston2013}}
  \label{table:costs}
\end{table}

